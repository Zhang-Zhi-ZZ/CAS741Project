\documentclass{article}

\usepackage{tabularx}
\usepackage{booktabs}
\usepackage{comment}
\usepackage{amsmath}


\title{CAS 741: Problem Statement\\Double Pendulum}

\author{Zhi Zhang\\400005778}
\date{22/19/2019}

\begin{document}
\maketitle
\begin{table}[hp]
\caption{Revision History} \label{TblRevisionHistory}
\begin{tabularx}{\textwidth}{llX}
\toprule
\textbf{Date} & \textbf{Developer(s)} & \textbf{Change}\\
\midrule
18/09/2019 & Zhi Zhang & Initial Draft\\
22/09/2019 & Zhi Zhang & Changed project from polynomial interpolation to double pendulum\\
02/10/2019 & Zhi Zhang & Fix date\\
\bottomrule
\end{tabularx}
\end{table}

\section{Problem}
A double pendulum consists of two pendulums attached end to end. With simply two limbs and two masses, a double pendulum can perform various motions. The moving curve of a double pendulum is highly sensitive to the initial conditions, which makes it a chaotic system, and the fundamentals of its motion gives full expression to dynamics, physics and mathematics. Studying the motions of a double pendulum can help us better understand the deterministic feature of a chaotic system.\\\\
In this project, we restrict the motion of a double pendulum in two dimensions. Considering a double pendulum with masses $\mathit{m_{1}}$ and $\mathit{m_{2}}$, and set the wires attaching them have lengths $\mathit{l_{1}}$ and $\mathit{l_{2}}$. Let $\mathit{\theta_{1}}$ and $\mathit{\theta_{2}}$ be the angles of the two wires, we will first get the potential and kinetic energy equations of the system, and use them to get the Lagrangian of the system which is the difference between the two. Then use the derived Lagrangian equation, we will determine the equations of the movements of the system, and demonstrate the double pendulum's strong sensitivity to initial conditions. 
 
\section{Context of Problem}
The stakeholders of this project would be Dr.Smith and all students taking CAS 741. And the environment for this software is MacOS.
 

\end{document}