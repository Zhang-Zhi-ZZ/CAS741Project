\documentclass[12pt, titlepage]{article}

\usepackage{booktabs}
\usepackage{tabularx}
\usepackage{hyperref}
\hypersetup{
    colorlinks,
    citecolor=green,
    filecolor=black,
    linkcolor=red,
    urlcolor=blue
}
\usepackage[round]{natbib}

%% Comments

\usepackage{color}

\newif\ifcomments\commentstrue

\ifcomments
\newcommand{\authornote}[3]{\textcolor{#1}{[#3 ---#2]}}
\newcommand{\todo}[1]{\textcolor{red}{[TODO: #1]}}
\else
\newcommand{\authornote}[3]{}
\newcommand{\todo}[1]{}
\fi

\newcommand{\wss}[1]{\authornote{blue}{SS}{#1}} 
\newcommand{\plt}[1]{\authornote{magenta}{TPLT}{#1}} %For explanation of the template
\newcommand{\an}[1]{\authornote{cyan}{Author}{#1}}


\begin{document}

\title{Test Report: Double Pendulum} 
\author{Zhi Zhang}
\date{\today}
  
\maketitle

\pagenumbering{roman}

\section{Revision History}

\begin{tabularx}{\textwidth}{p{3cm}p{2cm}X}
\toprule {\bf Date} & {\bf Version} & {\bf Notes}\\
\midrule
Dec.17 & 1.0 & Final Draft\\
 
\bottomrule
\end{tabularx}

~\newpage

\section{Symbols, Abbreviations and Acronyms}

\renewcommand{\arraystretch}{1.2}
\begin{tabular}{l l} 
  \toprule    
  \textbf{symbol} & \textbf{description}\\
  \midrule 
  T & Test\\
  \bottomrule
\end{tabular}\\

\newpage

\tableofcontents

\listoftables %if appropriate

\listoffigures %if appropriate

\newpage

\pagenumbering{arabic}

This document introduces the result of the system VnV test. 

\section{Functional Requirements Evaluation}
All the functional requirements have been met. 

\section{Nonfunctional Requirements Evaluation}
Generally, all the nonfunctional requirements have been met.

\subsection{Usability}
The system is easy to use, anyone with general computer technology is able to use it. Tested with 5 users, all of them get the output with their own input data. 

\subsection{Correctness and Verifiability}
The outputs generated by Double Pendulum were compared to the $\theta_1$ and $\theta_2$ graph from \url{https://www.myphysicslab.com/pendulum/double-pendulum-en.html} with the same input data, and the graphs match well. 

\subsection{Maintainability}
The source code was examined by the developer, and ensured that each module only performs one function, makes it easy to maintain.  


\subsection{Portability}
The unit testing has been performed on Mac OS X 10.11 and Windows 10. All functions works on both systems.
  
\section{Comparison to Existing Implementation} 
Compared to the existing implementation\url{https://www.myphysicslab.com/pendulum/double-pendulum-en.html}, Double Pendulum generated the list of $\theta_1(t)$ and $\theta_2(t)$ results, but does not provide animation of the motions, and does not provide as many graphs as the existing implementation\url{https://www.myphysicslab.com/pendulum/double-pendulum-en.html}. 

\section{Unit Testing}
The Unit testing has been done with Unittest in accordance with Unit VnV. All test succeeds. 

\section{Changes Due to Testing}
No change has been made. 

\section{Automated Testing}
    
\section{Trace to Requirements}
    
\section{Trace to Modules}    

\section{Code Coverage Metrics}

\bibliographystyle{plainnat}

\bibliography{SRS}

\end{document}