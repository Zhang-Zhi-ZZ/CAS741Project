\documentclass[12pt, titlepage]{article}

\usepackage{booktabs}
\usepackage{tabularx}
\usepackage{hyperref}
\usepackage{amsmath, mathtools}
\usepackage{amsfonts}
\usepackage{amssymb}
\usepackage{graphicx}
\usepackage{colortbl}

\usepackage{longtable}
\usepackage{xfrac}
\usepackage{tabularx}
\usepackage{float}
\usepackage{siunitx}
\usepackage{booktabs}
\usepackage{caption}
\usepackage{pdflscape}
\usepackage{afterpage}
\usepackage{cool}
\usepackage{comment}
\usepackage{adjustbox}
\usepackage{xr}
\usepackage{hyperref}
\hypersetup{
    colorlinks,
    citecolor=green,
    filecolor=black,
    linkcolor=red,
    urlcolor=green
}
\usepackage[round]{natbib}

%% Comments

\usepackage{color}

\newif\ifcomments\commentstrue

\ifcomments
\newcommand{\authornote}[3]{\textcolor{#1}{[#3 ---#2]}}
\newcommand{\todo}[1]{\textcolor{red}{[TODO: #1]}}
\else
\newcommand{\authornote}[3]{}
\newcommand{\todo}[1]{}
\fi

\newcommand{\wss}[1]{\authornote{blue}{SS}{#1}} 
\newcommand{\plt}[1]{\authornote{magenta}{TPLT}{#1}} %For explanation of the template
\newcommand{\an}[1]{\authornote{cyan}{Author}{#1}}

%% Common Parts

\newcommand{\progname}{ProgName} % PUT YOUR PROGRAM NAME HERE %Every program
                                % should have a name

\externaldocument{../../Design/MG/MG}
\externaldocument{../../SRS/SRS}

\newcounter{mnum}
\newcommand{\mthemnum}{M\themnum}
\newcommand{\mref}[1]{M\ref{#1}}



\begin{document}

\title{Unit Verification and Validation Plan for Double Pendulum} 
\author{Zhi Zhang}
\date{\today}
	
\maketitle

\pagenumbering{roman}

\section{Revision History}

\begin{tabularx}{\textwidth}{p{3cm}p{2cm}X}
\toprule {\bf Date} & {\bf Version} & {\bf Notes}\\
\midrule
Dec.3, 2019 & 1.0 & Initial Draft\\
\bottomrule
\end{tabularx}

~\newpage

\tableofcontents

%\listoftables

%\wss{Do not include if not relevant}

%\listoffigures

%\wss{Do not include if not relevant}

\newpage

\section{Symbols, Abbreviations and Acronyms}

\renewcommand{\arraystretch}{1.2}
\begin{tabular}{l l} 
  \toprule		
  \textbf{symbol} & \textbf{description}\\
  \midrule 
  T & Test\\
  VnV & Verification and Validation\\
  DP & Double Pendulum\\
  \bottomrule
\end{tabular}\\

%\wss{symbols, abbreviations or acronyms -- you can reference the SRS, MG or MIS
 % tables if needed}

\newpage

\pagenumbering{arabic}

This document provides an overview of the unit Verification and Validation(VnV) for Double Pendulum. It  lays out the purpose, methods, and test cases for the VnV procedure. 
% \wss{provide an introductory blurb and roadmap of the
 % unit V\&V plan}

\section{General Information}

\subsection{Purpose}
The purpose of this document is to describe the unit VnV plan of the software Double Pendulum(DP). This software gets input parameters from users to initialize the double pendulum system, then predict the motion of the system using ODE initial value problem solvers, output the result and displays the graph of the motions.   
%\wss{Identify software that is being unit tested (verified).}

\subsection{Scope}
The scope of DP is limited to the generate the output files and plot the diagram of change in the angles for two rods.
The scope of the test plan is described below:
\begin{itemize}
  \item Double Pendulum will be written in Python.
  \item The proposed test plan is focusing on system and unit testing to verify the functional and nonfunctional requirements of DP. 
\end{itemize}

The DP is limited to the user initialized inputs and output of the DP will be a text file and the generated graphs of the angular position of two pendulums.  

The modules that might be tested for under Unit VnV plan that are traced to the SRS requirements are: 

\begin{description}
\item [\mref{mIP}:] Input Parameters Module
\item [\mref{mOF}:] Output Format Module
\item [\mref{mAE}:] Acceleration Equations Module
\item [\mref{mC}:] Control Module
\item [\mref{mSDS}:] Sequence Data Structure Module
\item [\mref{mODE}:] ODE Solver Module
\item [\mref{mP}:] Plotting Module
 

\end{description}


\begin{comment}
\wss{What modules are outside of the scope.  If there are modules that are
  developed by someone else, then you would say here if you aren't planning on
  verifying them.  There may also be modules that are part of your software, but
  have a lower priority for verification than others.  If this is the case,
  explain your rationale for the ranking of module importance.}
\end{comment}
\section{Plan}
	
\subsection{Verification and Validation Team}

\begin{itemize}
  \item Zhi Zhang

\end{itemize}

\subsection{Automated Testing and Verification Tools}

Unit-based scripts can be created for testing the modules in DP and be tested automatically with scripts written in pytest. The test scripts using pytest will cycle through an array of data and check an assert statement.  
\subsection{Non-Testing Based Verification}
The non-testing based verification will involve a code inspection by myself during final submission. Dr. Smith will run the software to ensure the functional requirements are implemented.  
 
\section{Unit Test Description}
The modules that are being unit tested are specified in the MIS(\url{https://github.com/best-zhang-zhi/CAS741Project/blob/master/Double%20Pendulum/docs/Design/MIS/MIS.pdf}). The Unit VnV plan ensures that each module is behaving accurately and that each module is satisfying the SRS(\url{https://github.com/karolserkis/CAS-741-Pendula/blob/master/docs/SRS/SRS.pdf}) requirements. 
 

\subsection{Tests for Functional Requirements}

\subsubsection{Input Parameters Module} \label{test_input}
The following tests are created to ensure that the input parameters module is validate the user input data. This first step is critical as all of the other modules rely on these parameters.   
 
\begin{enumerate}

\item{test-Input\\}

Type: Automatic
					
Initial State: DP is opened
					
Input: DP import the input file
					
Output: assert=True

Test Case Derivation: The data inputted by the user should match the state variable from InParams.

How test will be performed: Random number values ranging from 1 to 100 will be inputed in the environment. This value should be exactly equal to the state variables of Input. An assert statement will return True if these are equal. 


\item{test-Input\\}

Type: Automatic
          
Initial State: DP is opened
          
Input: DP import the input file with invalid input data
          
Output: assert=True

Test Case Derivation: The system should detect the invalid data, and raise exception.  

How test will be performed: Invalid inputs will be inputed in the environment. The system should raise the correct exception. 

    
\end{enumerate}

\subsubsection{Acceleration Equations Module} \label{test_acceleration}
 
\begin{enumerate}

\item{test-Acceleration\\}

Type: Manual
          
Initial State: DP is opened
          
Input: Acceleration module is called
          
Output: equations of $\theta_1(t)$ and $\theta_2(t)$ 

Test Case Derivation: Output should match the equations calculated by hand 

How test will be performed: Different set of valid input will be passed to acceleration module, and the test team will check if the generated equations match the hand calculated equations. 
          
\end{enumerate}


\subsubsection{ODE Solver Module}\label{test_ODE}
\begin{enumerate}

\item{test-ODE\\}

Type: Automatic
          
Initial State: DP is opened
          
Input: ODE module called
          
Output: list of $\theta_1(t)$ and $\theta_2(t)$ data generated

Test Case Derivation: $\theta_1$ and $\theta_2$ values should be generated, and will be compared with the results generated by MATLAB with the same input data.   

How test will be performed: Unit test will call the function with valid inputs, and check if the output data matches the result from MATLAB.
          
\end{enumerate}

\subsubsection{Output Module} \label{test_output}
\begin{enumerate}

\item{test-Control\\}

Type: Manual
          
Initial State: DP is opened
          
Input: main Module is called
          
Output: output.txt file generated  

Test Case Derivation: Output should contain a list of $\theta_1(t)$ and $\theta_2(t)$ data.

How test will be performed: Unit test will call the function with valid inputs, and check if the output.txt file is generated in the project folder. 
          
\end{enumerate}

\subsubsection{Plotting Module} \label{test_plot}
\begin{enumerate}

\item{test-Plot\\}

Type: Automatic
          
Initial State: DP is opened
          
Input: Plot module called
          
Output: Graph of $\theta_1(t)$ and $\theta_2(t)$ displayed

Test Case Derivation: The graph be the plots of $\theta_1$ and $\theta_2$ values 

How test will be performed: Unit test will call the function with the inputs, and check if the output files are generated.
\end{enumerate}    

\subsection{Tests for Nonfunctional Requirements}
Testing for nonfunctional requirements is not a part of unit test, it is introduced in System VnV Plan(\url{https://github.com/best-zhang-zhi/CAS741Project/blob/master/Double%20Pendulum/docs/VnVPlan/SystVnVPlan/SystVnVPlan.pdf})
 
\subsection{Traceability Between Test Cases and Modules}

The purpose of the traceability matrices is to provide easy references on what
has to be additionally modified if a certain component is changed. Table
\ref{Tb_trace} shows the
dependencies between the test cases and the requirements.

\begin{table}[h]
\centering
\begin{tabular}{@{}ll@{}}
\toprule
Test Cases & Modules \\ \midrule
\ref{test_input} & Input Parameters Module \\
\ref{test_acceleration} & Acceleration Equations Module\\
\ref{test_ODE} & ODE Solver Module \\
\ref{test_output} & Output Format Module \\
\ref{test_plot} & Plotting Module\\
\bottomrule
\end{tabular}
\caption{Traceability Matrix showing the connections between modules and
tests}
\label{Tb_trace}
\end{table}


\bibliographystyle{plainnat}

\bibliography{SRS}

\newpage

\section{Appendix}

%\wss{This is where you can place additional information, as appropriate}

\subsection{Symbolic Parameters}
There is no symbolic parameters for Double Pendulum. 
 
\end{document}