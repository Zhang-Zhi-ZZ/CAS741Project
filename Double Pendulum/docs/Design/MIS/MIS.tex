\documentclass[12pt, titlepage]{article}

\usepackage{amsmath, mathtools}

\usepackage[round]{natbib}
\usepackage{amsfonts}
\usepackage{amssymb}
\usepackage{graphicx}
\usepackage{colortbl}
\usepackage{xr}
\usepackage{hyperref}
\usepackage{longtable}
\usepackage{xfrac}
\usepackage{tabularx}
\usepackage{float}
\usepackage{siunitx}
\usepackage{booktabs}
\usepackage{multirow}
\usepackage[section]{placeins}
\usepackage{caption}
\usepackage{fullpage}
\usepackage{mathtools}
\usepackage{comment}

\hypersetup{
bookmarks=true,     % show bookmarks bar?
colorlinks=true,       % false: boxed links; true: colored links
linkcolor=red,          % color of internal links (change box color with linkbordercolor)
citecolor=blue,      % color of links to bibliography
filecolor=magenta,  % color of file links
urlcolor=cyan          % color of external links
}

\usepackage{array}

\externaldocument{../../SRS/SRS}

%% Comments

\usepackage{color}

\newif\ifcomments\commentstrue

\ifcomments
\newcommand{\authornote}[3]{\textcolor{#1}{[#3 ---#2]}}
\newcommand{\todo}[1]{\textcolor{red}{[TODO: #1]}}
\else
\newcommand{\authornote}[3]{}
\newcommand{\todo}[1]{}
\fi

\newcommand{\wss}[1]{\authornote{blue}{SS}{#1}} 
\newcommand{\plt}[1]{\authornote{magenta}{TPLT}{#1}} %For explanation of the template
\newcommand{\an}[1]{\authornote{cyan}{Author}{#1}}


\newcommand{\progname}{Double Pendulum}

\begin{document}

\title{Module Interface Specification for Double Pendulum}

\author{Zhi Zhang}

\date{\today}

\maketitle

\pagenumbering{roman}

\section{Revision History}

\begin{tabularx}{\textwidth}{p{3cm}p{2cm}X}
\toprule {\bf Date} & {\bf Version} & {\bf Notes}\\
\midrule
Nov.13 & 1.0 & Initial Draft\\
\bottomrule
\end{tabularx}

~\newpage

\section{Symbols, Abbreviations and Acronyms}

See SRS Documentation at
\url{https://github.com/best-zhang-zhi/CAS741Project/blob/master/Double%20Pendulum/docs/SRS/SRS.pdf}

\newpage

\tableofcontents

\newpage

\pagenumbering{arabic}

\section{Introduction}

The following document details the Module Interface Specifications for Double
Pendulum, a software which determines the motion of a double pendulum given the
initial conditions from user inputs.

Complementary documents include the System Requirement Specifications
and Module Guide.  The full documentation and implementation can be
found at \url{https://github.com/best-zhang-zhi/CAS741Project}.  

\section{Notation}

The structure of the MIS for modules comes from \citet{HoffmanAndStrooper1995},
with the addition that template modules have been adapted from
\cite{GhezziEtAl2003}.  The mathematical notation comes from Chapter 3 of
\citet{HoffmanAndStrooper1995}.  For instance, the symbol := is used for a
multiple assignment statement and conditional rules follow the form $(c_1
\Rightarrow r_1 | c_2 \Rightarrow r_2 | ... | c_n \Rightarrow r_n )$.

The following table summarizes the primitive data types used by \progname. 

\begin{center}
\renewcommand{\arraystretch}{1.2}
\noindent 
\begin{tabular}{l l p{7.5cm}} 
\toprule 
\textbf{Data Type} & \textbf{Notation} & \textbf{Description}\\ 
\midrule
character & char & a single symbol or digit\\
integer & $\mathbb{Z}$ & a number without a fractional component in (-$\infty$, $\infty$) \\
natural number & $\mathbb{N}$ & a number without a fractional component in [1, $\infty$) \\
real & $\mathbb{R}$ & any number in (-$\infty$, $\infty$)\\
string & $char^n$ & a sequence of alphanumeric and special characters\\
list & $real^n$ & a list of real numbers\\
\bottomrule
\end{tabular} 
\end{center}

\noindent
The specification of \progname \ uses some derived data types: sequences, strings, and
tuples. Sequences are lists filled with elements of the same data type. Strings
are sequences of characters. Tuples contain a list of values, potentially of
different types. In addition, \progname \ uses functions, which
are defined by the data types of their inputs and outputs. Local functions are
described by giving their type signature followed by their specification.

\section{Module Decomposition}

The following table is taken directly from the Module Guide document for this project.

\begin{table}[h!]
\centering
\begin{tabular}{p{0.3\textwidth} p{0.6\textwidth}}
\toprule
\textbf{Level 1} & \textbf{Level 2}\\
\midrule

{Hardware-Hiding Module} & ~ \\
\midrule

\multirow{7}{0.3\textwidth}{Behaviour-Hiding Module} 
& Input Parameters Module\\
& Output Format Module\\
& Acceleration Equations Module\\
& Velocity ODEs Module\\
& Control Module\\

\midrule

\multirow{3}{0.3\textwidth}{Software Decision Module} 
& Sequence Data Structure Module\\
& ODE Solver Module\\
& Plotting Module\\
\bottomrule

\end{tabular}
\caption{Module Hierarchy}
\label{TblMH}
\end{table}

\newpage
~\newpage

\section{MIS of Control Module} \label{CModule} 
The control module provides the main program. 

\subsection{Module}

main

\subsection{Uses}
Param (Section~\ref{IPModule}), Acceleration (Section~\ref{AEModule}),
Velocity (Section~\ref{VOModule}), Solver (Section~\ref{ODEModule}),
Plot (Section~\ref{PModule}), Output (Section~\ref{OModule})

\subsection{Syntax}

\subsubsection{Exported Constants}

N/A
\subsubsection{Exported Access Programs}

\begin{center}
\begin{tabular}{p{4cm} p{2cm} p{2cm} p{6cm}}
\hline
\textbf{Name} & \textbf{In} & \textbf{Out} & \textbf{Exceptions} \\
\hline
main & - & - &  \\
\hline
\end{tabular}
\end{center}

\subsection{Semantics}

\subsubsection{State Variables}
N/A

\subsubsection{Environment Variables}
N/A

\subsubsection{Assumptions}
\begin{itemize}
  \item Users only input numerical numbers
\end{itemize}

\subsubsection{Access Routine Semantics}

\noindent main():
\begin{itemize}
\item transition: Modify the state of Param module and the environment variables for the Plot and Output modules by following steps\\
\end{itemize}

\noindent Get(filenameIn: String) and (filenameOut: string) from user\\

\noindent load\_params(filenameIn)\\
\noindent \#\textit{Find angular velocity function} ($\theta_1, \theta_2$)\\
\noindent ${\theta_1}'$:= \text{solve}(\text{Acceleration\_Top},0.0, 0.0,$t$)
\wss{Where is the value of $t$ given?}\\
\noindent ${\theta_2}'$:= \text{solve}(\text{Acceleration\_Bottom},0.0, 0.0,$t$)\\
\noindent \#\textit{Find angular position function} ($\theta_1, \theta_2$)\\
\noindent $\theta_1$:= \text{solve}(\text{Velocity\_Top},0.0, 0.0,$t$)\\
\noindent $\theta_2$:= \text{solve}(\text{Velocity\_Bottom},0.0, 0.0,$t$)\\

\wss{You have a second order ODE, but looking at your document, it appears that
  you haven't really thought about how to solve it.  The SWHS example that you
  based your document on is for a first order ODE.  You'll need to make changes.
  Not big changes, but changes.  For instance, your solver modules shows the
  equation for solving a first order ODE.  I think the easiest thing for you to
  do is to focus on what you actually need.  You need theta as a function of
  time.  You don't really need the angular velocity.  Therefore, your solver
  should take the definition of theta'' and return theta.  You can leave the
  angular velocity out of it altogether.  Your solver module then just has to
  say mathematically what it means to solve a second order ODE.}

\noindent \#\textit{Output calculated values to a file and to a plot.}\\

\noindent plot($\theta_1$, $\theta_2$)\\

\noindent output(filenameOut, $\theta_1$, $\theta_2$)\\


\subsubsection{Local Functions}
N/A

\newpage

\section{MIS of Input Parameter Module} \label{IPModule} The secrets of this
module are the data structure for input parameters, how the values are input and
how the values are verified. The load and verify secrets are isolated to their
own access programs.

\subsection{Module}

Param

\subsection{Uses}
N/A
\subsection{Syntax}

\subsubsection{Exported Constants}

N/A

\subsubsection{Exported Access Programs}

\begin{center}
\begin{tabular}{p{4cm} p{2cm} p{2cm} p{6cm}}
\hline
\textbf{Name} & \textbf{In} & \textbf{Out} & \textbf{Exceptions} \\
\hline
load\_params & string & - & FileError \\
\hline
verify\_params & - & - & NEGATIVE\_MASS, NEGATIVE\_LENGTH, NEGATIVE\_GRAVITY, NEGATIVE\_TIME\\
\hline
$m_1$ & - & $\mathbb{R}$ & - \\
\hline
$m_2$ & - & $\mathbb{R}$ & -\\
\hline
$L_1$ & - & $\mathbb{R}$ & - \\
\hline
$L_2$ & - & $\mathbb{R}$ & -\\
\hline
$\theta_1$ & - & $\mathbb{R}$ & -\\
\hline
$\theta_2$ & - & $\mathbb{R}$ & - \\
\hline
$g$ & - & $\mathbb{R}$ & - \\
\hline
$t$ & - & $\mathbb{R}$ & - \\
\hline
\end{tabular}
\end{center}

\subsection{Semantics}

\subsubsection{State Variables}
\renewcommand{\arraystretch}{1.2}
\begin{longtable*}[l]{l} 

$m_1$: $\mathbb{R}$ \\
$m_2$: $\mathbb{R}$ \\
$L_1$: $\mathbb{R}$ \\
$L_2$: $\mathbb{R}$ \\
$\theta_1$: $\mathbb{R}$ \\
$\theta_2$: $\mathbb{R}$ \\
$g$ : $\mathbb{R}$ \\
$t$: $\mathbb{R}$ \\

\end{longtable*}

\subsubsection{Environment Variables}
inputFile: sequence of string \#\textit{f[i] is the ith string in the text file f}\\ 
\subsubsection{Assumptions}
\begin{itemize}
\item load\_params will be called before the values of any state variables will be accessed.

\item The file contains the string equivalents of the numeric values for
each input parameter in order, each on a new line. The order is the same as in
the table in R1 of the SRS. Any comments in the input file should be denoted
with a '\#' symbol.
\end{itemize}
\subsubsection{Access Routine Semantics}

\noindent Param.$m_1$:
\begin{itemize}
\item transition: N/A
\item output: $out := m_1$
\item exception: none
\end{itemize}

\noindent Param.$m_2$:
\begin{itemize}
\item transition: N/A
\item output: $out := m_2$
\item exception: none
\end{itemize}

\noindent Param.$L_1$:
\begin{itemize}
\item transition: N/A
\item output: $out := L_1$
\item exception: none
\end{itemize}

\noindent Param.$L_2$:
\begin{itemize}
\item transition: N/A
\item output: $out := L_2$
\item exception: none
\end{itemize}

\noindent Param.$\theta_1$:
\begin{itemize}
\item transition: N/A
\item output: $out := \theta_1$
\item exception: N/A
\end{itemize}

\noindent Param.$\theta_2$:
\begin{itemize}
\item transition: N/A
\item output: $out := \theta_2$
\item exception: N/A
\end{itemize}


\noindent Param.$g$:
\begin{itemize}
\item transition: N/A
\item output: $out := g$
\item exception: none
\end{itemize}

\noindent Param.$t$:
\begin{itemize}
\item transition: N/A
\item output: $out := t$
\item exception: none
\end{itemize}

\noindent load\_params($s$):
\begin{itemize}
\item transition: The filename $s$ is first associated with the file f.  {inputFile} is used to
  modify the state variables using the following procedural specification:
\begin{enumerate}
\item Read data sequentially from inputFile to populate the state variables from
  R1 ($m_1$ to $\mathit{t}$).


\item verify\_params()


\end{enumerate}
\item output: N/A
\item exception: exc := a file name $s$ cannot be found OR the format of
  inputFile is incorrect $\Rightarrow$  FileError
\end{itemize}

\noindent verify\_params():
\begin{itemize}
\item transition: N/A
\item out: \textit{out} := none
\item exception: exc := 
\end{itemize}
\noindent \begin{longtable*}[l]{l l} 
$\neg (m_1 > 0)$ & $\Rightarrow$ NEGATIVE\_MASS\\
$\neg (m_2 > 0)$ & $\Rightarrow$ NEGATIVE\_MASS\\
$\neg (L_1 > 0)$ & $\Rightarrow$ NEGATIVE\_LENGTH\\
$\neg (L_2 > 0)$ & $\Rightarrow$ NEGATIVE\_LENGTH\\
$\neg (g > 0)$ & $\Rightarrow$ NEGATIVE\_GRAVITY\\
$\neg (t > 0)$ & $\Rightarrow$ NEGATIVE\_TIME\\
\end{longtable*}

\subsubsection{Local Functions}
N/A

\newpage


\section{MIS of Acceleration Equations Module} \label{AEModule} 
 
\subsection{Module}
Acceleration

\subsection{Uses}
Input

\subsection{Syntax}

\subsubsection{Exported Constants}
N/A

\subsubsection{Exported Access Programs}

\begin{center}
\begin{tabular}{p{4cm} p{2cm} p{2cm} p{4cm}}
\hline
\textbf{Name} & \textbf{In} & \textbf{Out} & \textbf{Exceptions} \\
\hline
Acceleration\_Top & - & $\mathbb{R}$ & - \\
\hline
Acceleration\_Bottom & - & $\mathbb{R}$ & - \\
\hline
\end{tabular}
\end{center}

\wss{You probably do not want to split the acceleration into two access
  programs.  You are solving a system of ODEs, so the ODE solver will expect
  both inputs (top and bottom).  You are better off outputting a sequence of two
  values.}

\subsection{Semantics}

\subsubsection{State Variables}
\begin{itemize}
\item ${\theta_1}''$
\item ${\theta_2}''$
\end{itemize}

\subsubsection{Environment Variables}
N/A
\subsubsection{Assumptions}
N/A
\subsubsection{Access Routine Semantics}

\noindent Acceleration\_Top():
\begin{itemize}
\item transition: \[{\theta_1}''=\frac{-g(2m_1+m_2)sin\theta_1-m_2gsin(\theta_1-2\theta_2)-2sin(\theta_1-\theta_2)m_2({{\theta_2}'}^2L_2+{{\theta_2}'}^2L_1cos(\theta_1-\theta_2)\big)}{L_1(2m_1+m_2-m_2cos(2\theta_1-2\theta_2))}\] 
\item output: N/A
\item exception: N/A
\end{itemize}

\noindent Acceleration\_Bottom():
\begin{itemize}
\item transition: \[{\theta_2}''=\frac{2sin(\theta_1-\theta_2)({\theta_1}'L_1(m1+m2)+g(m_1+m_2)cos(\theta_1)+{(\theta_2}')^2L_2m_2cos(\theta_1-\theta_2\big)}{L_2(2m_1+m_2-m_2cos(2\theta_1-2\theta_2))}\] 
\item output: N/A
\item exception: N/A
\end{itemize}

\subsubsection{Local Functions}

N/A
\newpage


\section{MIS of Velocity ODEs Module} \label{VOModule} 
 
\subsection{Module}
Velocity \wss{As mentioned in the first module, I don't think you need the
  Velocity module.}

\subsection{Uses}
Acceleration

\subsection{Syntax}

\subsubsection{Exported Constants}
N/A

\subsubsection{Exported Access Programs}

\begin{center}
\begin{tabular}{p{4cm} p{2cm} p{2cm} p{4cm}}
\hline
\textbf{Name} & \textbf{In} & \textbf{Out} & \textbf{Exceptions} \\
\hline
Velocity\_Top & - & $\mathbb{R} \rightarrow \mathbb{R}^2$ & - \\
\hline
Velocity\_Bottom & - & $\mathbb{R} \rightarrow \mathbb{R}^2$ & - \\
\hline
\end{tabular}
\end{center}

\subsection{Semantics}

\subsubsection{State Variables}
N/A
\subsubsection{Environment Variables}
N/A
\subsubsection{Assumptions}
N/A
\subsubsection{Access Routine Semantics}

\noindent Velocity\_Top():
\begin{itemize}
\item transition: N/A 
\item output: $out:= $$\iint {\theta_1}'' \,d{\theta_1}'\,d{\theta_2}'$$ $
\item exception: N/A
\end{itemize}

\noindent Velocity\_Bottom():
\begin{itemize}
\item transition: N/A 
\item output: $out:= $$\iint {\theta_2}'' \,d{\theta_1}'\,d{\theta_2}'$$ $
\item exception: N/A
\end{itemize}

\subsubsection{Local Functions}

N/A
\newpage

\begin{comment}
\section{MIS of Position ODEs Module} \label{POModule} 
 
\subsection{Module}
Velocity

\subsection{Uses}
Velocity

\subsection{Syntax}

\subsubsection{Exported Constants}
N/A

\subsubsection{Exported Access Programs}

\begin{center}
\begin{tabular}{p{4cm} p{2cm} p{2cm} p{4cm}}
\hline
\textbf{Name} & \textbf{In} & \textbf{Out} & \textbf{Exceptions} \\
\hline
Position\_Top & - & $\mathbb{R}^2 \rightarrow \mathbb{R}^3$ & - \\
\hline
Position\_Bottom & - & $\mathbb{R}^2 \rightarrow \mathbb{R}^3$ & - \\
\hline
\end{tabular}
\end{center}

\subsection{Semantics}

\subsubsection{State Variables}
N/A
\subsubsection{Environment Variables}
N/A
\subsubsection{Assumptions}
N/A
\subsubsection{Access Routine Semantics}

\noindent Position\_Top():
\begin{itemize}
\item transition: N/A 
\item output: $out:= $$\iint {\theta_1}' \,d{\theta_1}\,d{\theta_2}$$ $
\item exception: N/A
\end{itemize}

\noindent Position\_Bottom():
\begin{itemize}
\item transition: N/A 
\item output: $out:= $$\iint {\theta_2}' \,d{\theta_1}\,d{\theta_2}$$ $
\item exception: N/A
\end{itemize}

\subsubsection{Local Functions}

N/A

\newpage
\end{comment}

\section{MIS of ODE Solver Module} \label{ODEModule} 
\#\textit{Bold font is used to indicate variables that are a sequence
  type} 
\subsection{Module}
Solver($n: \mathbb{N}$) \#\textit{$n$ is the length of the sequences}

\subsection{Uses}
None

\subsection{Syntax}

\subsubsection{Exported Constants}
N/A
\subsubsection{Exported Access Programs}

\begin{center}
\begin{tabular}{p{1.5cm} >{\raggedright\arraybackslash}p{9.25cm} >{\raggedright\arraybackslash}p{2.4cm} p{1.5cm}}
  \hline
  \textbf{Name} & \textbf{In} & \textbf{Out} & \textbf{Except.} \\
  \hline
  solve & $\textbf{f}: (\mathbb{R}^{n} \rightarrow \mathbb{R}^{n+1}), t_0 : \mathbb{R},
          \textbf{y}_0: \mathbb{R}^n, 
          t_\text{fin}: \mathbb{R}$ & $t_1: \mathbb{R}, 
                                      \textbf{y}:
                                      (\mathbb{R}
                                      \rightarrow
                                      \mathbb{R})^{n+1}$ & -\\

  \hline 
\end{tabular}
\end{center}
\subsection{Semantics}

\subsubsection{State Variables}
N/A

\subsubsection{Environment Variables}

N/A
\subsubsection{Assumptions}

N/A
\subsubsection{Access Routine Semantics}

\#\textit{Solving} $ $$\iint {y} $$ = \mathbf{f}(t,
\mathbf{y}(t))$\\

\noindent solve($\textbf{f}, t_0, \textbf{y}_0, t_\text{fin}$): 
\begin{itemize}
\item output: $out := t_1, \textbf{y}(t)$ where 
$$\textbf{y}(t) = \textbf{y}_0 + \int_{t_0}^{t} \textbf{f}(s, \textbf{y}(s)) ds$$ 
with $t_1$ determined to be 0.05 second.  $\textbf{y}(t)$ is
calculated from $t = t_0$ to $t = t_1$.
\item exception: N/A
\end{itemize}
\subsubsection{Local Functions}

N/A

\newpage
\section{MIS of Plotting Module} \label{PModule} 

\subsection{Module}
Plot
\subsection{Uses}
N/A
\subsection{Syntax}

\subsubsection{Exported Constants}

\subsubsection{Exported Access Programs}
\begin{center}
\begin{tabular}{p{2cm} p{8cm} p{2cm} p{2cm}}
\hline
\textbf{Name} & \textbf{In} & \textbf{Out} & \textbf{Exceptions} \\
\hline
plot & $\theta_1(t):\mathbb{R} \rightarrow \mathbb{R},
                 \theta_2(t):\mathbb{R} \rightarrow \mathbb{R}
                 $ & - & - \\
\hline
\end{tabular}
\end{center}

\subsection{Semantics}

\subsubsection{State Variables}
N/A

\subsubsection{Environment Variables}

win: 2D sequence of pixels displayed on the screen\\

\subsubsection{Assumptions}
N/A
\subsubsection{Access Routine Semantics}

\noindent plot($\theta_1$, $\theta_2$):
\begin{itemize}
\item transition: Modify win to display a plot where the vertical axis
  is angular position and horizontal axis is time. The time should run from $0$ to $t$.
\item output: N/A
\item exception: N/A
\end{itemize}

\subsubsection{Local Functions}

N/A
\newpage


\section{MIS of Output Module} \label{OModule} 

\subsection{Module}

Output

\subsection{Uses}
Param(Section~\ref{IPModule})

\subsection{Syntax}

\subsubsection{Exported Constants}
N/A
\subsubsection{Exported Access Programs}

\begin{center}
\begin{tabular}{p{3cm} p{7cm} p{2cm} p{2cm}}
\hline
\textbf{Name} & \textbf{In} & \textbf{Out} & \textbf{Exceptions} \\
\hline
output & fname: string, $\theta_1(t):\mathbb{R} \rightarrow \mathbb{R},
                 \theta_2(t):\mathbb{R} \rightarrow \mathbb{R}$ & - & - \\
\hline
\end{tabular}
\end{center}

\subsection{Semantics}

\subsubsection{State Variables}
N/A

\subsubsection{Environment Variables}

file: A text file
\subsubsection{Assumptions}

N/A
\subsubsection{Access Routine Semantics}

\noindent output(fname, $\theta_1$, $\theta_2$.$t$):
\begin{itemize}
\item transition:  Write to environment variable named fname the
  following: the input
    parameters from Param, and the calculated values $\theta_1$, $\theta_2$from times $0$ to $t$.  The functions will be output as
    sequences in this file.  The spacing between points in the sequence should
    be selected so that the motion pf the pendulums is captured in the data.
\item output: N/A
\item exception: N/A
\end{itemize}

\subsubsection{Local Functions}

N/A
\newpage

 

\bibliographystyle {plainnat}
\bibliography {../../../refs/References}

\newpage

\section{Appendix} \label{Appendix}
\renewcommand{\arraystretch}{1.2}

\begin{longtable}{l p{12cm}}
\caption{Possible Exceptions} \\
\toprule
\textbf{Message ID} & \textbf{Error Message} \\
\midrule
FileError & Error: The inputed file is invalid\\
NEGATIVE\_MASS & Error: Mass of the pendulum must be $> 0$ \\
NEGATIVE\_LENGTH & Error: Length of rods must be $> 0$ \\
NEGATIVE\_GRAVITY & Error: Gravity must be $> 0$\\
NEGATIVE\_TIME& Error: Time for the simulation must be $> 0$ \\

\bottomrule
\end{longtable}

\end{document}