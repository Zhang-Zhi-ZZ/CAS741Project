\documentclass[12pt, titlepage]{article}

\usepackage{amsmath, mathtools}

\usepackage[round]{natbib}
\usepackage{amsfonts}
\usepackage{amssymb}
\usepackage{graphicx}
\usepackage{colortbl}
\usepackage{xr}
\usepackage{hyperref}
\usepackage{longtable}
\usepackage{xfrac}
\usepackage{tabularx}
\usepackage{float}
\usepackage{siunitx}
\usepackage{booktabs}
\usepackage{multirow}
\usepackage[section]{placeins}
\usepackage{caption}
\usepackage{fullpage}
\usepackage{mathtools}

\hypersetup{
bookmarks=true,     % show bookmarks bar?
colorlinks=true,       % false: boxed links; true: colored links
linkcolor=red,          % color of internal links (change box color with linkbordercolor)
citecolor=blue,      % color of links to bibliography
filecolor=magenta,  % color of file links
urlcolor=cyan          % color of external links
}

\usepackage{array}

\externaldocument{../../SRS/SRS}

\input{../../Comments}

\newcommand{\progname}{Double Pendulum}

\begin{document}

\title{Module Interface Specification for Double Pendulum}

\author{Zhi Zhang}

\date{\today}

\maketitle

\pagenumbering{roman}

\section{Revision History}

\begin{tabularx}{\textwidth}{p{3cm}p{2cm}X}
\toprule {\bf Date} & {\bf Version} & {\bf Notes}\\
\midrule
Nov.13 & 1.0 & Initial Draft\\
\bottomrule
\end{tabularx}

~\newpage

\section{Symbols, Abbreviations and Acronyms}

See SRS Documentation at \url{https://github.com/best-zhang-zhi/CAS741Project/blob/master/Double%20Pendulum/docs/SRS/SRS.pdf}

\newpage

\tableofcontents

\newpage

\pagenumbering{arabic}

\section{Introduction}

The following document details the Module Interface Specifications for Double Pendulum, a software which determines the motion of a double pendulum given the initial conditions from user inputs. 

Complementary documents include the System Requirement Specifications
and Module Guide.  The full documentation and implementation can be
found at \url{https://github.com/best-zhang-zhi/CAS741Project}.  

\section{Notation}

The structure of the MIS for modules comes from \citet{HoffmanAndStrooper1995},
with the addition that template modules have been adapted from
\cite{GhezziEtAl2003}.  The mathematical notation comes from Chapter 3 of
\citet{HoffmanAndStrooper1995}.  For instance, the symbol := is used for a
multiple assignment statement and conditional rules follow the form $(c_1
\Rightarrow r_1 | c_2 \Rightarrow r_2 | ... | c_n \Rightarrow r_n )$.

The following table summarizes the primitive data types used by \progname. 

\begin{center}
\renewcommand{\arraystretch}{1.2}
\noindent 
\begin{tabular}{l l p{7.5cm}} 
\toprule 
\textbf{Data Type} & \textbf{Notation} & \textbf{Description}\\ 
\midrule
character & char & a single symbol or digit\\
integer & $\mathbb{Z}$ & a number without a fractional component in (-$\infty$, $\infty$) \\
natural number & $\mathbb{N}$ & a number without a fractional component in [1, $\infty$) \\
real & $\mathbb{R}$ & any number in (-$\infty$, $\infty$)\\
string & $char^n$ & a sequence of alphanumeric and special characters\\
list & $real^n$ & a list of real numbers\\
\bottomrule
\end{tabular} 
\end{center}

\noindent
The specification of \progname \ uses some derived data types: sequences, strings, and
tuples. Sequences are lists filled with elements of the same data type. Strings
are sequences of characters. Tuples contain a list of values, potentially of
different types. In addition, \progname \ uses functions, which
are defined by the data types of their inputs and outputs. Local functions are
described by giving their type signature followed by their specification.

\section{Module Decomposition}

The following table is taken directly from the Module Guide document for this project.

\begin{table}[h!]
\centering
\begin{tabular}{p{0.3\textwidth} p{0.6\textwidth}}
\toprule
\textbf{Level 1} & \textbf{Level 2}\\
\midrule

{Hardware-Hiding Module} & ~ \\
\midrule

\multirow{7}{0.3\textwidth}{Behaviour-Hiding Module} 
& Input Parameters Module\\
& Output Format Module\\
& Velocity Equations Module\\
& Position ODEs Module\\
& Control Module\\

\midrule

\multirow{3}{0.3\textwidth}{Software Decision Module} 
& Sequence Data Structure Module\\
& ODE Solver Module\\
& Plotting Module\\
\bottomrule

\end{tabular}
\caption{Module Hierarchy}
\label{TblMH}
\end{table}

\newpage
~\newpage

\section{MIS of Control Module} \label{CModule} 
The control module provides the main program. 

\subsection{Module}

main

\subsection{Uses}
\wss{add later-zz}

\subsection{Syntax}

\subsubsection{Exported Constants}

N/A
\subsubsection{Exported Access Programs}

\begin{center}
\begin{tabular}{p{4cm} p{2cm} p{2cm} p{6cm}}
\hline
\textbf{Name} & \textbf{In} & \textbf{Out} & \textbf{Exceptions} \\
\hline
main & - & - &  \\
\hline
\end{tabular}
\end{center}

\subsection{Semantics}

\subsubsection{State Variables}
N/A

\subsubsection{Environment Variables}
N/A

\subsubsection{Assumptions}
\begin{itemize}
  \item Users only input numerical numbers
\end{itemize}

\subsubsection{Access Routine Semantics}

\noindent main():
\begin{itemize}
\item transition: Modify the state of Param module and the environment variables for the Plot and Output modules by following steps\\
\end{itemize}

\noindent Get(filenameIn: String) and (filenameOut: string) from user\\

\noindent load\_params(filenameIn)\\

\noindent \#\textit{Find angular position function} ($\theta_1, \theta_2$)\\


\noindent $\theta_1$:= \text{solve}(\text{ODE\_Position}, 0.0)\\

\wss{add later-zz}





\subsubsection{Local Functions}
N/A

\newpage

\section{MIS of Input Parameter Module} \label{IPModule} 
The secrets of this module are the data structure for input parameters, how the values are input and how the values are verified. The load and verify secrets are isolated to their own access programs.

\subsection{Module}

Param

\subsection{Uses}
SpecParam(Section\wss{add later})

\subsection{Syntax}

\subsubsection{Exported Constants}

N/A

\subsubsection{Exported Access Programs}

\begin{center}
\begin{tabular}{p{4cm} p{2cm} p{2cm} p{6cm}}
\hline
\textbf{Name} & \textbf{In} & \textbf{Out} & \textbf{Exceptions} \\
\hline
load\_params & string & - & FileError \\
\hline
verify\_params & - & - & NEGATIVE\_MASS, NEGATIVE\_LENGTH, NEGATIVE\_GRAVITY, NEGATIVE\_TIME\\
\hline
$m_1$ & - & $\mathbb{R}$ & - \\
\hline
$m_2$ & - & $\mathbb{R}$ & -\\
\hline
$L_1$ & - & $\mathbb{R}$ & - \\
\hline
$L_2$ & - & $\mathbb{R}$ & -\\
\hline
$\theta_1$ & - & $\mathbb{R}$ & -\\
\hline
$\theta_2$ & - & $\mathbb{R}$ & - \\
\hline
$g$ & - & $\mathbb{R}$ & - \\
\hline
$t$ & - & $\mathbb{R}$ & - \\
\hline
\end{tabular}
\end{center}

\subsection{Semantics}

\subsubsection{State Variables}
\renewcommand{\arraystretch}{1.2}
\begin{longtable*}[l]{l} 

$m_1$: $\mathbb{R}$ \\
$m_2$: $\mathbb{R}$ \\
$L_1$: $\mathbb{R}$ \\
$L_2$: $\mathbb{R}$ \\
$\theta_1$: $\mathbb{R}$ \\
$\theta_2$: $\mathbb{R}$ \\
$g$ : $\mathbb{R}$ \\
$t$: $\mathbb{R}$ \\

\end{longtable*}

\subsubsection{Environment Variables}
inputFile: sequence of string \#\textit{f[i] is the ith string in the text file f}\\ 
\subsubsection{Assumptions}
\begin{itemize}
\item load\_params will be called before the values of any state variables will be accessed.

\item The file contains the string equivalents of the numeric values for
each input parameter in order, each on a new line. The order is the same as in
the table in R1 of the SRS. Any comments in the input file should be denoted
with a '\#' symbol.
\end{itemize}
\subsubsection{Access Routine Semantics}

\noindent Param.$m_1$:
\begin{itemize}
\item transition: N/A
\item output: $out := m_1$
\item exception: none
\end{itemize}

\noindent Param.$m_2$:
\begin{itemize}
\item transition: N/A
\item output: $out := m_2$
\item exception: none
\end{itemize}

\noindent Param.$L_1$:
\begin{itemize}
\item transition: N/A
\item output: $out := L_1$
\item exception: none
\end{itemize}

\noindent Param.$L_2$:
\begin{itemize}
\item transition: N/A
\item output: $out := L_2$
\item exception: none
\end{itemize}

\noindent Param.$\theta_1$:
\begin{itemize}
\item transition: N/A
\item output: $out := \theta_1$
\item exception: N/A
\end{itemize}

\noindent Param.$\theta_2$:
\begin{itemize}
\item transition: N/A
\item output: $out := \theta_2$
\item exception: N/A
\end{itemize}


\noindent Param.$g$:
\begin{itemize}
\item transition: N/A
\item output: $out := g$
\item exception: none
\end{itemize}

\noindent Param.$t$:
\begin{itemize}
\item transition: N/A
\item output: $out := t$
\item exception: none
\end{itemize}

\noindent load\_params($s$):
\begin{itemize}
\item transition: The filename $s$ is first associated with the file f.  {inputFile} is used to
  modify the state variables using the following procedural specification:
\begin{enumerate}
\item Read data sequentially from inputFile to populate the state variables from
  R1 ($m_1$ to $\mathit{t}$).
\item Calculate the derived quantities (all other state variables) as follows:
\begin{itemize}
\item \[{\theta_1}''=\frac{f-g}{h}\]\newline
  where \[f = -g(2m_1+m_2)sin\theta_1-m_2gsin(\theta_1-2\theta_2)^2L_1cos(d)))\]\newline
  \[g = 2(sin(\theta_1-\theta_2)m_2({{\theta_2}'}^2L_2+{{\theta_2}'}^2L_1cos(d)\big)\]
  \[h = L_1(2m_1+m_2-m_2cos(2\theta_1-2\theta_2))\]
  where \[d = \theta_1-\theta_2\]
\item \[{\theta_2}''=\frac{2sin(d)({\theta_1}'L_1(m1+m2)+g(m_1+m_2)cos(\theta_1)+{(\theta_2}')^2L_2m_2cos(d)}{L_2(2m_1+m_2-m_2cos(2\theta_1-2\theta_2))}\]
  where \[d = \theta_1-\theta_2\]

\end{itemize}
\item verify\_params()


\end{enumerate}
\item output: N/A
\item exception: exc := a file name $s$ cannot be found OR the format of
  inputFile is incorrect $\Rightarrow$  FileError
\end{itemize}

\noindent verify\_params():
\begin{itemize}
\item transition: N/A
\item out: \textit{out} := none
\item exception: exc := 
\end{itemize}
\noindent \begin{longtable*}[l]{l l} 
$\neg (m_1 > 0)$ & $\Rightarrow$ NEGATIVE\_MASS\\
$\neg (m_2 > 0)$ & $\Rightarrow$ NEGATIVE\_MASS\\
$\neg (L_1 > 0)$ & $\Rightarrow$ NEGATIVE\_LENGTH\\
$\neg (L_2 > 0)$ & $\Rightarrow$ NEGATIVE\_LENGTH\\
$\neg (g > 0)$ & $\Rightarrow$ NEGATIVE\_GRAVITY\\
$\neg (T> 0)$ & $\Rightarrow$ NEGATIVE\_TIME\\
\end{longtable*}

\subsubsection{Local Functions}
N/A

\newpage


\section{MIS of Position ODEs Module} \label{POModule} 
 
\subsection{Module}
Acceleration

\subsection{Uses}
Input

\subsection{Syntax}

\subsubsection{Exported Constants}
\begin{itemize}
  \item ${\theta_1}''$
  \item ${\theta_2}''$
\end{itemize}

\subsubsection{Exported Access Programs}

\begin{center}
\begin{tabular}{p{4cm} p{2cm} p{2cm} p{4cm}}
\hline
\textbf{Name} & \textbf{In} & \textbf{Out} & \textbf{Exceptions} \\
\hline
getAcc1 & Input & string & - \\
\hline
getAcc2 & Input & string & - \\
\hline
\end{tabular}
\end{center}

\subsection{Semantics}

\subsubsection{State Variables}
N/A
\subsubsection{Environment Variables}
N/A
\subsubsection{Assumptions}
N/A
\subsubsection{Access Routine Semantics}

\noindent getAcc1($m_1$, $m_2$, $L_1$. $L_2$, $\theta_1$, $\theta_2$, $g$):
\begin{itemize}
\item transition: N/A 
\item output: 
\[{\theta_1}''=\frac{-g(2m_1+m_2)sin\theta_1-m_2gsin(\theta_1-2\theta_2)-2sin(\theta_1-\theta_2)m_2({{\theta_2}'}^2L_2+{{\theta_2}'}^2L_1cos(\theta_1-\theta_2)\big)}{L_1(2m_1+m_2-m_2cos(2\theta_1-2\theta_2))}\]
\item exception: N/A
\end{itemize}

\noindent getAcc2($m_1$, $m_2$, $L_1$. $L_2$, $\theta_1$, $\theta_2$, $g$):
\begin{itemize}
\item transition: N/A 
\item output: 
\[{\theta_2}''=\frac{2sin(\theta_1-\theta_2)({\theta_1}'L_1(m1+m2)+g(m_1+m_2)cos(\theta_1)+{(\theta_2}')^2L_2m_2cos(\theta_1-\theta_2\big)}{L_2(2m_1+m_2-m_2cos(2\theta_1-2\theta_2))}\]
\item exception: N/A
\end{itemize}

\subsubsection{Local Functions}

N/A
\newpage
\section{MIS of ODE Solver Module} \label{ODEModule} 
Multi-variable Rung-kutta algorithm will be used to approximate the angular velocity of the two mass. \cite{RungeKutta} This module outputs two lists of point velocity for each mass. 
\subsection{Module}
RungeKutta
\subsection{Uses}
Acceleration

\subsection{Syntax}

\subsubsection{Exported Constants}
\begin{itemize}
  \item ${\theta_1}'$
  \item ${\theta_2}'$
\end{itemize}
\subsubsection{Exported Access Programs}

\begin{center}
\begin{tabular}{p{2cm} p{4cm} p{4cm} p{2cm}}
\hline
\textbf{Name} & \textbf{In} & \textbf{Out} & \textbf{Exceptions} \\
\hline
getVelocity1 & string & list & - \\
\hline
getVelocity2 & string & list & - \\
\hline
\end{tabular}
\end{center}

\subsection{Semantics}

\subsubsection{State Variables}
\begin{itemize}
  \item $a_n$ = ${\theta_1}''(0.1,{\theta_1}')$ 
  \item $b_n$ = ${\theta_1}''(0.105,{\theta_1}'+0.005a_n)$
  \item $c_n$ = ${\theta_1}''(0.105,{\theta_1}'+0.005b_n)$
  \item $d_n$ = ${\theta_1}''(0.11,{\theta_1}'+0.01b_n)$

\end{itemize}
\subsubsection{Environment Variables}

N/A
\subsubsection{Assumptions}

N/A
\subsubsection{Access Routine Semantics}

\noindent getVelocity1():
\begin{itemize}
\item transition: N/A 
\item output: $\forall n: \mathbb{Z}| n \in[0..10000]: \theta_1(n+1) = \theta_1(n) + 0.01/6(a_n+2b_n+2c_n+d_n) $
\item exception: N/A
\end{itemize}

\noindent getVelocity2():
\begin{itemize}
\item transition: N/A 
\item output: $\forall n: \mathbb{Z}| n \in[0..10000]: \theta_2(n+1) = \theta_2(n) + 0.01/6(a_n+2b_n+2c_n+d_n) $
\item exception: N/A
\end{itemize}

\subsubsection{Local Functions}

N/A
\newpage


\section{MIS of Velocity Equations Module} \label{VEModule} 
The output module takes the point velocity of two masses, and outputs the point position of them in list form.

\subsection{Module}
Output

\subsection{Uses}
RungeKutta

\subsection{Syntax}

\subsubsection{Exported Constants}
\subsubsection{Exported Constants}
\begin{itemize}
  \item ${\theta_1}$
  \item ${\theta_2}$
\end{itemize}

\subsubsection{Exported Access Programs}

\begin{center}
\begin{tabular}{p{2cm} p{4cm} p{4cm} p{2cm}}
\hline
\textbf{Name} & \textbf{In} & \textbf{Out} & \textbf{Exceptions} \\
\hline
getPosition1 & list & list & - \\
\hline
getPosition2 & list & list & - \\
\hline
\end{tabular}
\end{center}

\subsection{Semantics}

\subsubsection{State Variables}

N/A
\subsubsection{Environment Variables}
N/A
\subsubsection{Assumptions}
N/A
\subsubsection{Access Routine Semantics}

\noindent getPosition1($(\theta_1)_n,(\theta_1)_0$):
\begin{itemize}
\item transition: N/A  
\item output: $\forall n: \mathbb{Z}| n \in[1..9999]: \theta_1(n+1) = \theta_1(n-1) + \theta_1(n)$
\item exception: N/A
\end{itemize}

\subsubsection{Local Functions}

N/A

\newpage

\section{MIS of Output Module} \label{OModule} 


\wss{You can reference SRS labels, such as R\ref{R_Inputs}.}

\wss{It is also possible to use \LaTeX for hypperlinks to external documents.}

\subsection{Module}

Output

\subsection{Uses}
Param(Section~\ref{IPModule})

\subsection{Syntax}

\subsubsection{Exported Constants}
N/A
\subsubsection{Exported Access Programs}

\begin{center}
\begin{tabular}{p{2cm} p{4cm} p{4cm} p{2cm}}
\hline
\textbf{Name} & \textbf{In} & \textbf{Out} & \textbf{Exceptions} \\
\hline
\wss{accessProg} & - & - & - \\
\hline
\end{tabular}
\end{center}

\subsection{Semantics}

\subsubsection{State Variables}

\wss{Not all modules will have state variables.  State variables give the module
  a memory.}

\subsubsection{Environment Variables}

\wss{This section is not necessary for all modules.  Its purpose is to capture
  when the module has external interaction with the environment, such as for a
  device driver, screen interface, keyboard, file, etc.}

\subsubsection{Assumptions}

\wss{Try to minimize assumptions and anticipate programmer errors via
  exceptions, but for practical purposes assumptions are sometimes appropriate.}

\subsubsection{Access Routine Semantics}

\noindent \wss{accessProg}():
\begin{itemize}
\item transition: \wss{if appropriate} 
\item output: \wss{if appropriate} 
\item exception: \wss{if appropriate} 
\end{itemize}

\wss{A module without environment variables or state variables is unlikely to
  have a state transition.  In this case a state transition can only occur if
  the module is changing the state of another module.}

\wss{Modules rarely have both a transition and an output.  In most cases you
  will have one or the other.}

\subsubsection{Local Functions}

\wss{As appropriate} \wss{These functions are for the purpose of specification.
  They are not necessarily something that is going to be implemented
  explicitly.  Even if they are implemented, they are not exported; they only
  have local scope.}

\newpage
\section{MIS of Plotting Module} \label{PModule} 


\wss{You can reference SRS labels, such as R\ref{R_Inputs}.}

\wss{It is also possible to use \LaTeX for hypperlinks to external documents.}

\subsection{Module}

\wss{Short name for the module}

\subsection{Uses}


\subsection{Syntax}

\subsubsection{Exported Constants}

\subsubsection{Exported Access Programs}

\begin{center}
\begin{tabular}{p{2cm} p{4cm} p{4cm} p{2cm}}
\hline
\textbf{Name} & \textbf{In} & \textbf{Out} & \textbf{Exceptions} \\
\hline
\wss{accessProg} & - & - & - \\
\hline
\end{tabular}
\end{center}

\subsection{Semantics}

\subsubsection{State Variables}

\wss{Not all modules will have state variables.  State variables give the module
  a memory.}

\subsubsection{Environment Variables}

\wss{This section is not necessary for all modules.  Its purpose is to capture
  when the module has external interaction with the environment, such as for a
  device driver, screen interface, keyboard, file, etc.}

\subsubsection{Assumptions}

\wss{Try to minimize assumptions and anticipate programmer errors via
  exceptions, but for practical purposes assumptions are sometimes appropriate.}

\subsubsection{Access Routine Semantics}

\noindent \wss{accessProg}():
\begin{itemize}
\item transition: \wss{if appropriate} 
\item output: \wss{if appropriate} 
\item exception: \wss{if appropriate} 
\end{itemize}

\wss{A module without environment variables or state variables is unlikely to
  have a state transition.  In this case a state transition can only occur if
  the module is changing the state of another module.}

\wss{Modules rarely have both a transition and an output.  In most cases you
  will have one or the other.}

\subsubsection{Local Functions}

\wss{As appropriate} \wss{These functions are for the purpose of specification.
  They are not necessarily something that is going to be implemented
  explicitly.  Even if they are implemented, they are not exported; they only
  have local scope.}

\newpage


\newpage


\bibliographystyle {plainnat}
\bibliography {../../../refs/References}

\newpage

\section{Appendix} \label{Appendix}

\wss{Extra information if required}

\end{document}