\documentclass{article}

\usepackage{tabularx}
\usepackage{booktabs}
\usepackage{comment}
\usepackage{amsmath}


\title{CAS 741: Problem Statement\\Polynomial Interpolation}

\author{Zhi Zhang\\400005778}
\date{18/19/2019}

\begin{document}
\maketitle
\begin{table}[hp]
\caption{Revision History} \label{TblRevisionHistory}
\begin{tabularx}{\textwidth}{llX}
\toprule
\textbf{Date} & \textbf{Developer(s)} & \textbf{Change}\\
\midrule
18/19/2019 & Zhi Zhang & Initial Draft\\
\bottomrule
\end{tabularx}
\end{table}

\section{Problem}

Polynomial interpolation is a method to estimate values with a given data set. Given a set of $\mathit{n+1}$ data points $\mathit{(xi, yi)}$ where no two $\mathit{x_{i}}$ are the same, we want to compute a polynomial $\mathit{p_{n}}$ of degree at most $\mathit{n}$ such that 
\\\centerline{$\mathit{p_{n-1}(x_{i})=f_{i}, i  = 0,1, ...,n.}$}\\
\\Various methods can be used to construct a polynomial interpolation. In this project, we will introduce 3 methods: monomial, Newton and Lagrange.
\\
\\ In monomial basis, the polynomial will be generated in the form of: 
\\\centerline{$\mathit{p_{n}(x)=a_{0}+a_{1}x+...+a_{n-1}x^{n-1}+a_{n}x^{n}}$.}
\\In Newton basis, the polynomial will be generated in the form of:
\\\centerline{$\mathit{p_{n}(x)=\displaystyle \sum_{i=0}^{n} c_{i}n_{i}(x)=c_{0}+c_{1}(x-x_{0})+c_{2}(x-x_{0})(x-x_{1})+...+c_{n}\displaystyle \prod_{j=0}^{n-1} (x-x_{j})}$.} 
\\In Lagrange basis, the polynomial will be generated in the form of:\\
\\\centerline{$\mathit{L(x)= \sum_{j=0}^{k} y_{j}l_{j}(x)}$}\\ 
where $\mathit{l_{j}(x) = \displaystyle \prod_{0\leq m\leq k,m\neq j} \frac{x-x_{m}}{x_{j}-x_{m}}}$.

\section{Context of Problem}
The stakeholders of this project would be all students taking SFWR ENG 4X03 course. And the environment for this software is MacOS.
 

\end{document}